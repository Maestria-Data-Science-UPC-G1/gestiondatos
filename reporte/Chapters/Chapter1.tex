\chapter{Introducción} % Main chapter title
\label{Chapter1} % For referencing the chapter elsewhere, use \ref{Chapter1} 

%----------------------------------------------------------------------------------------

% Define some commands to keep the formatting separated from the content
% Placing such commands in the preamble is a good idea.
\newcommand{\keyword}[1]{\textbf{#1}}
\newcommand{\tabhead}[1]{\textbf{#1}}
\newcommand{\code}[1]{\texttt{#1}}
\newcommand{\file}[1]{\texttt{\bfseries#1}}
\newcommand{\option}[1]{\texttt{\itshape#1}}

%-

\section{Descripción de la información}

En el presente trabajo se está realizando ...

\section{Datos utilizados}

La información con la que se va a trabajar está conformada por los datos de alumnos de niveles inicial, primaria y secundaria de colegios en todo el Perú, esta información corresponde a los 3 últimos años, el detalle de esta entidad se ve en la tabla Table~\ref{tab:estudiantes}

\begin{table}
\caption{Detalle de los campos de la entidad estudiante}
\label{tab:estudiantes}
\centering
\begin{tabular}{l l l}
\toprule
\tabhead{Columna} & \tabhead{Descripción}\\
\midrule
ID Año & Identificador del año escolar\\
Departamento & Nombre del departamento\\
Provincia & Nombre de la provincia\\
Distrito & Nombre del distrito\\
Centro Poblado & Centro poblado \\
NLongitud IE & Coordenada de la longitud de la IE \\
NLatitud IE & Coordenada de la latitud de la IE\\
DRE & Dirección regional de educación\\
UGEL & Ugel a la que pertenece la IE\\
Cod Mod & NN\\
Anexo & NN\\
Nombre IE & Nombre de la Institución educativa\\
D Gestión & NN\\
Dsc Turno & Descripción del turno\\
Modalidad & Modalidad\\
Nivel Educativo & Nivel en el que se encuentra el estudiante\\
Fecha Matrícula & Fecha de la matrícula\\
ID Persona & Identificador del estudiante\\
Documento Identidad & Documento de identidad\\
Sexo & Sexo del alumno\\
Madre vive & Indicador si la madre vive\\
Padre vive & Indicador si el padre vive\\
Dsc Grado & Descripción del grado\\
Dsc Sección & Descripción de la sección\\
Nacionalidad & Nacionalidad del estudiante\\
Dsc Discapacidad & Descripción de discapacidad\\
Dirección & Dirección del estudiante\\
Lugar & Lugar de la dirección\\
Ubigeo & Ubigeo del estudiante\\
\bottomrule\\
\end{tabular}
\end{table}

\section{Importancia de los datos}

Debido a que la información trabajada proviene del ámbito educativo de níveles básicos, se pueden generar estadísticas para poder evaluar el estado actual educativo, ...