Download

Source

PDF
Actions
   Copy Project
   Word Count
Sync
   Dropbox
   Git
   GitHub
Settings
Compiler

pdfLaTeX
TeX Live version

2022
Main document

main.tex
Spell check

English
Dictionary
Auto-complete

On
Auto-close Brackets

On
Code check

On
Editor theme

overleaf
Overall theme

Default
Keybindings

None
Font Size

12px
Font Family

Lucida / Source Code Pro
Line Height

Normal
PDF Viewer

Overleaf
Help
   Show Hotkeys
   Documentation
   Contact Us
Gestion_de_datos

File outline
Overleaf has upgraded the source editor. You can still use the old editor by selecting "Source (legacy)".

Click to learn more and give feedback
Editor mode.


 
Selection deleted
▾
▾
▾
▾
▾
▾
% !TEX root = ../main.tex

%----------------------------------------------------------------------------------------
% TITLE PAGE
%----------------------------------------------------------------------------------------

\newgeometry{margin=1in}
\begin{titlepage}

% Make the title page mostly inert to the parskip-setting.
\setlength{\parskip}{0pt}

\begin{center}
\includegraphics[width=1\textwidth]{Figures/upc_p}

%\ifxetex
    \vspace{0.6cm}
    {\zhawtitlefont\color{zhawblue}\LARGE \univname\par}   % University
    \vspace{0.2cm}
%\else
%    \vspace{0.87cm}
%    {\includegraphics[height=17.9pt]{Figures/zhaw_font_eng_font}\par}
%    \vspace{0.05cm}
%\fi
{\Large  \deptname\par}                      % Department
\vspace{0.2cm}
%{\Large A \instname\par}                                 % Institute
\vspace{3.5cm}                            
\textsc{\Large \ttype}                                 % Thesis type
\vspace{0.2cm}
\HRule 
\vspace{0.4cm}
{\huge \bfseries \ttitle\par}                          % Thesis title
\vspace{0.4cm}  
\HRule
\vspace{1.5cm}

 
\begin{minipage}[t]{0.4\textwidth}
\begin{flushleft} 
    \large
    \emph{Integrantes:}\\
    \authorname
\end{flushleft}
\end{minipage}
\begin{minipage}[t]{0.4\textwidth}
\begin{flushright} 
    \large
    \emph{Profesor:} \\
    \supnameA \\
    \supnameB
\end{flushright}
\end{minipage}
\vspace{2cm}
 
\vfill

{\large
Fecha de entrega\\
28 de Octubre de 2022 \\
\vspace{1.5cm}
%Study program:\\
%\studyprog\\
}
\vfill
\end{center}
\end{titlepage}
\restoregeometry

Wieman
0 of 0
×
Replace with
UNIVERSIDAD PERUANA DE CIENCIAS APLICADAS
Maestría en Data Science
GESTIÓN DE DATOS
Análisis de matrícula de estudiantes de nivel
inicial de Educación Básica Regular del 2020,
2021 y 2022
Integrantes:
Alexander Cubas Francia
Ernesto Igor Laura Mamani
Grabiela Quispe Ochoa
Yerimen Antonio Campos Luyo
Profesor:
Oscar Efraín Ramos Ponce
Fecha de entrega
28 de Octubre de 2022
iii
Índice
1 Introducción 1
1.1 Descripción de la información . . . . . . . . . . . . . . . . . . . . . . . . 1
1.2 Datos utilizados . . . . . . . . . . . . . . . . . . . . . . . . . . . . . . . . 1
1.3 Importancia de los datos . . . . . . . . . . . . . . . . . . . . . . . . . . . 1
2 Introduction to LATEX 3
2.1 Filling in Your Information in the main.tex File . . . . . . . . . . . . . 3
2.1.1 References . . . . . . . . . . . . . . . . . . . . . . . . . . . . . . . 3
3 Metodología 5
3.1 Metodología utilizada . . . . . . . . . . . . . . . . . . . . . . . . . . . . 5
3.2 Herramientas usadas . . . . . . . . . . . . . . . . . . . . . . . . . . . . . 5
3.3 Trabajo con los datos . . . . . . . . . . . . . . . . . . . . . . . . . . . . . 6
3.4 Uso de shiny . . . . . . . . . . . . . . . . . . . . . . . . . . . . . . . . . . 6
4 Resultados 7
4.1 Evolución mensual de matrículas . . . . . . . . . . . . . . . . . . . . . . 7
5 Anexos 11
5.1 Código de conexión a la base de datos . . . . . . . . . . . . . . . . . . . 11
5.2 Evolución mensual de matrículas . . . . . . . . . . . . . . . . . . . . . . 11
Bibliografía 13
1
Capítulo 1
Introducción
1.1 Descripción de la información
En el presente trabajo se está realizando ...
1.2 Datos utilizados
La información con la que se va a trabajar está conformada por los datos de alumnos
de niveles inicial, primaria y secundaria de colegios en todo el Perú, esta información
corresponde a los 3 últimos años, el detalle de esta entidad se ve en la tabla Table 1.1
1.3 Importancia de los datos
Debido a que la información trabajada proviene del ámbito educativo de níveles
básicos, se pueden generar estadísticas para poder evaluar el estado actual educa-
tivo, ...
2 Capítulo 1. Introducción
TABLA 1.1: Detalle de los campos de la entidad estudiante
Columna Descripción
ID Año Identificador del año escolar
Departamento Nombre del departamento
Provincia Nombre de la provincia
Distrito Nombre del distrito
Centro Poblado Centro poblado
NLongitud IE Coordenada de la longitud de la IE
NLatitud IE Coordenada de la latitud de la IE
DRE Dirección regional de educación
UGEL Ugel a la que pertenece la IE
Cod Mod NN
Anexo NN
Nombre IE Nombre de la Institución educativa
D Gestión NN
Dsc Turno Descripción del turno
Modalidad Modalidad
Nivel Educativo Nivel en el que se encuentra el estudiante
Fecha Matrícula Fecha de la matrícula
ID Persona Identificador del estudiante
Documento Identidad Documento de identidad
Sexo Sexo del alumno
Madre vive Indicador si la madre vive
Padre vive Indicador si el padre vive
Dsc Grado Descripción del grado
Dsc Sección Descripción de la sección
Nacionalidad Nacionalidad del estudiante
Dsc Discapacidad Descripción de discapacidad
Dirección Dirección del estudiante
Lugar Lugar de la dirección
Ubigeo Ubigeo del estudiante
3
Capítulo 2
Introduction to LATEX
2.1 Filling in Your Information in the main.tex File
2.1.1 References
The options used in the main.tex file mean that the in-text citations of references
are formatted with the author(s) listed with the date of the publication. Multiple
references are separated by semicolons (e.g. [1, 2]) and references with more than
three authors only show the first author with et al. indicating there are more authors
(e.g. [3]). This is done automatically for you. To see how you use references, have
a look at the Chapter1.tex source file. Many reference managers allow you to
simply drag the reference into the document as you type.
5
Capítulo 3
Metodología
3.1 Metodología utilizada
En el presente proyecto se ha utilizado la metología CRISP DM, pero sólo en las
etapas iniciales que tienen que ver con el trabajo de los datos. En la imagen Figure 3.1
se pueden ver resaltadas las etapas cubiertas.
IMAGEN 3.1: Secciones cubiertas en el proyecto
3.2 Herramientas usadas
Se han hecho uso de las siguientes herramientas:
• Base de datos MySql
• Servicio RDS de AWS
• R Studio
• JetBrains DataGrip
• Shiny
6 Capítulo 3. Metodología
Se obtuvo un archivo CSV con toda la información de estudiantes, este archivo fue
subido mediante DataGrip a una BD MySQL, previamente creada como un servi-
cio RDS de AWS, con estos datos cargados se realizaron procesos mediante R para
obtener información que posteriormente fueron registrados en otra base de datos y
mostrados mediante una aplicación Shiny.
En la imagen Figure 3.2 se puede ver el flujo del proceso en este trabajo de análisis.
IMAGEN 3.2: Herramientas usadas
3.3 Trabajo con los datos
Para el trabajo con R y MySQL se hizo uso de la bilbioteca RMySQL y DBI, el código
para la conexión se encuentra en la sección 5.1.
3.4 Uso de shiny
...
